\documentclass[a4paper,12pt]{book}
\usepackage[utf8x]{inputenc}
\usepackage{amssymb, amsmath}
\setcounter{chapter}{1}
\newcommand{\ubar}{\underbar}
\usepackage{fullpage}

\author{Studenti vari}
\title{Schema per il primo compitino}

\begin{document}

\maketitle

\section{Preambolo}
\paragraph{Unicità del problema di Cauchy}
$$\dot{x}\ddot{x}$$

\section{Equazioni del moto}

Un sistema dinamico in generale potrà essere descritto da una forma del tipo:
$$\bigg\{ \begin{array}{l}
\dot{x} = f(x, y)\\
\dot{y} = g(x, y)\\
\end{array}$$

\subsection{Diagrammi di fase}
\paragraph{Punti di equilibrio} Analiticamente:
$$f(x, y) = 0$$ $$g(x, y) = 0$$
Sono loro attorno ai quali possono avvenire le orbite (??).
\paragraph{Orbite periodiche}
Analiticamente:
$$ ( x(t+\tau), y(t+\tau) ) = (x(t), y(t))$$
Sono curve chiuse sulla quale il campo vettoriale non si annulla.
Per il teorema di esistenza e unicità non possono esservi allora orbite che la intersechino.
\paragraph{Orbite asintotiche a punti di equilibrio}
``Salvano" l'unicità della soluzione.
\paragraph{Orbite aperte non asintotiche}
Il punto che descrive lo stato del sistema le percore senza intersecare altre orbite e senza essere stato in un punto già visto.
\paragraph{Costante del moto}
È una funzione $\Phi$ tale che:
$$\Phi(x(t), y(t)) = \Phi(x_0, y_0)\ \ \forall t$$

NOTA: se $\Phi \in \mathbb{C}^1$, allora $\Phi$ è una costante del moto per il sistema
$$\bigg\{ \begin{array}{l}
\dot{x} = f(x, y)\\
\dot{y} = g(x, y)\\
\end{array}$$
se e solo se
$$f\left(\dfrac{\partial \Phi}{\partial x}\right) + g\left(\dfrac{\partial \Phi}{\partial y}\right) = 0$$
Questa proposizione serve perché a questo punto sappiamo che per schematizzare il moto basta tracciare le curve di livello identificandone eventuali singolarità. Ogni curva di livello è un insieme di orbite.

\subparagraph{Esempio - Equazioni di Newton}
$$\Biggl\{ \begin{array}{l}
\dot{x} = v\\
\dot{y} = \dfrac{F(x)}{m}\\
\end{array}$$
Flusso (???).
Tutti i punti di equilibrio sull'asse $x$ hanno $v=0$.
Punti dell'asse x non di equilibio hanno orbite a tangente verticale.

\paragraph{Sistemi conservativi}
Un sistema è conservativo se $F(x)$ ammette primitiva $U(x)$ tale che $$F(x) = -\dfrac{\partial U}{\partial x}$$ da cui $$m\ddot{x} + U'(x) = 0 \ \ \forall x$$
Per i sistemi conservativi descritti dalle equazioni di Newton l'energia è una costante del moto:
$$ E = \dfrac{1}{2} mv^2+U(x)$$
Fissata una certa energia $E_0$ è possibile il moto solo per $E\geq U(x) = E_0$ ovvero $$E-U(x)=E-E_0 = T \geq 0$$
I casi possibili sono:
\begin{itemize}
	\item $\exists x_0: \forall x < x_0,\ U(x) < E_0$ oltre che $\forall x > x_0,\ U(x) > E_0$: allora il punto proviene dall'infinito e ci ritorna; si ha v=0 quando $U(\bar{x}) = E_0$ ma $\dot{v}(\bar{x})$ non è zero, quindi l'accelerazione non è nulla
	\item Intervallo chiuso se $U(x) \leq E_0 \Rightarrow x \in (a, b) \subset \mathbb{R}$ allora si ha un'orbita periodica chiusa con estremi in cui si inverte il moto.
	\item Nota: se $a$ o $b$ sono dei punti di massimo relativi di $U(x)$, allora si hanno dei punti singolari: vedi la sezione sulle separatrici subito sotto.
\end{itemize}
\paragraph{Buche di potenziale}
Sono i punti in cui $U(x)$ ha un minimo
\paragraph{Separatrici}
In presenza di massimi di $U(x)$, il diagramma di fase è simmetrico rispetto a $x$: $E$ dipende quadraticamente da $v$, e quindi $E(x, v) = E(x, -v)$.
In particolare, ci saranno due rette che separeranno il diagramma di coefficiente angolare:
$$y'(x_0) = \pm\sqrt{U''(x_0)}$$

\subsection{Osservazioni sui diagrammi di fase}
Se $U(x) \to E_0$ e $U$ allora $y \to 0$, e se $E_0$ non è un massimo relativo di $U(x)$, si ha una tangente verticale sul diagramma di fase.


\subsection{Moti periodici}
\paragraph{Periodo di oscillazione}
Il periodo di oscillazione (andata e ritorno) tra $x_1$ e $x_2$\footnote{dove $x_1$ e $x_2$ sono i punti sono della radici di $U(x) = E$}, è dato dalla relazione
$$ T(E) = \sqrt{2m}\int_{x_1(E)}^{x_2(E)} \dfrac{dx}{\sqrt{E-U(x)}}$$

\subsection{Come risolvere un esercizio}
Facciamo innanzi tutto qualche considerazione di base. Per prima cosa, ricordiamo che:
$$y = \pm \sqrt{2(E_0-U(x)}$$



\subsection{Ricorda}


\section{Meccanica Lagrangiana}
\subsection{Definizioni principali}
La funzione di Lagrange è la funzione:
$$ L = T - U $$
e soddisfa la relazione
$${d \over dt}\dfrac{\partial L}{\partial\dot{q}_i} = \dfrac{\partial L}{\partial q_i}$$
\subsection{Come risolvere un esercizio}

\subsection{Ricorda}
\paragraph{Teorema di Koenig}

\paragraph{Teorema di Huygens-Steiner}
Ricorda: non puoi usarlo se: (nota di lorenzoni)

\section{Altro}
\paragraph{Regola di Cramer}
\paragraph{Autovalori e autovettori}

\end{document}


